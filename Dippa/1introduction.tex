\chapter{Introduction}
\label{chapter:intro}

% Software engineering is at a crossroads as there are new leaner and more
% agile software development methods appearing next to the traditional
% software development methods.
\mika{Kappaleen pointti: No literature reviews of actual metric use}
Software metrics have been studied for decades and several literature reviews
have been published.
Yet, the literature reviews have been written from an academic viewpoint that
typically focuses on the effectiveness of a single metric. For example, Catal
et al. review fault prediction metrics \citeppri{catal2009systematic}, Purao
et al. review metrics for object oriented systems \citepri{purao2003product}
and Kitchenham performs a mapping of most cited software metrics
papers \citepri{kitchenham_whats_2010}. To our knowledge there are no
systematic literature reviews on the actual use of software metrics in the industry.

\mika{Kappaleen pointti: Agile on t�rke�� eik� metriikkoja tutkittu}

\juha{Yritin konkretisoida trad vs. agile kontrastia}
Agile software development is becoming increasing popular in the software
industry. The agile approach seems to be contradicting with the traditional
metrics approaches. For example, the agile emphasizes working software over measuring progress in terms of intermediate products or documentation, and embracing the change invalidates the traditional approach of tracking progress against pre-made plan.
However, at the same time agile software development
highlights some measures that should be used, e.g., burndowqn graphs and 100\%
automated unit testing coverage. However, measurement research in the context of agile
methods remains scarce.

The goal of this paper is to review the literature of actual use of software
metrics in the context of agile software development. This study will lay out
the current state of metrics usage in industrial agile software development
based on literature.
Moreover, the study uncovers the reasons for metric usage as well as
highlights actions that the use of metrics can trigger. Due to our research
goal, we focus this paper on case studies and actual empirical findings
excluding theoretical discussion and models lacking empirical validation. 
% The performed SR has more research questions but for this paper only
In this paper we, cover the following research questions:

\begin{enumerate}
  \item Why are metrics used?
  \item What actions do the use of metrics trigger?
  \item Which metrics are used?
\end{enumerate}

\begin{comment}
This paper is structured as follows. \Cref{sec:Method} describes how
the SR was conducted. \Cref{sec:Results} reports the results from the study.
\Cref{sec:Discussion} discusses about the findings and how they map to agile
principles. \Cref{sec:Conclusions} concludes the paper and suggests next
steps.
\end{comment}


