\chapter{Introduction}
\label{chapter:intro}
 MMASDMADSMAMDAMASDASDM


\section{Problem statement}

 

\section{Structure of the Thesis}
\label{section:structure} 

This thesis is organized as follows\ldots

\subsection{Important metrics}
 
This section describes metrics that were considered important.

Progress as working code was considered as one of the cornerstones of agile
\citepri{Trimble20134826}.

Capacity as number of features developed in release was considered better than
measuring speed, since speed is generally thought as a attribute of humans.
Capacity on the otherhand measures the capabilities of an organization.

Story flow percentage and velocity of elaborating features were considered as
key metrics for monitoring projects. Also, a minimum 60\% value for flow was
identified. Similarly, velocity for elaborating features should be as fast as
velocity of implementing features. Also, they said using both metrics
\emph{``drive behaviors to let teams go twice as fast as they could before''}. \citepri{Jakobsen2011168}

Net Promoter Score (NPV) was said to be \emph{``one of the purest measures of
success''} ~\citepri{Green2011}.

Story percent complete metric was considered valuable since it embraces test
driven development - no progress is made before test is written. Also, percent
complete metric is considered more accurate than previous metric. Moreover, it
gives normalized measure of progress compared to developer comments about
progress. Additionally, story percent complete metric leverages existing unit
testing framework and thus requires only minimal overhead to track progress.
Team members seemed to be extremely happy about using the metric.
\citepri{Trapa2006243}

Pseudo-velocity was considered essential for release planning
\citepri{Polk2011263}.

Burndown was valuable in meeting sprint commitments \citepri{Green2011}.
Similary, managers said burndown was important in making decisions and
managing multiple teams \citepri{Dubinsky200512}. However, developers didn't
consider burndown important \citepri{Dubinsky200512}.

Top teams at Adobe estimated backlog items with relative effort estimates
\citepri{Green2011}.

Practitioners at Ericsson valued transparency and overview of progress that
the metrics were able to provide to the complex product development with
parallel activities, namely cost types, rate of requirements over phases and
variance in handovers \citepri{Petersen2011975}.








