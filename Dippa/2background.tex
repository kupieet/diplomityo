\chapter{Background}
\label{chapter:background} 

The what problem must have some background, otherwise it is not
interesting.  You can explain the background here. Probably you should
change the title to something that describes more the content of this
chapter. Background consists of information that help other masters of
the same degree program to understand the rest of the thesis.

Transitions mentioned in Section~\ref{section:structure} are used also
in the chapters and sections. For example, next in this chapter we
tell how to use English language, how to find and refer to sources,
and enlight different ways to include graphics in the thesis.

\section{Language and Structure}

Moreover, the transitions are also used in the paragraph and the
sentence level meaning that all the text is linked together. For example,
the word ``moreover'' here is one way, but of course you should use
variation in the text. Examples of transitional devices (words) and
their use can be found from writing guides, e.g. from the Online
Writing Lab
(OWL)\footnote{http://owl.english.purdue.edu/owl/resource/574/02/} of
Purdue University or Strunk's Elements of
Style\footnote{http://www.bartleby.com/141/}. Remember that footnotes
are additional information, and they are seldom used.  If you refer to a source, you do no
not use footnote. The right command for the references is \emph{cite}.

The language used in the thesis should be technical (or
scientifical). For example, the abreviations aren't used but all them
are written open (i.e. ``are not''). Since the content itself is often
hard to understand (and explain), the sentences should not be very
long, use complex language with several examples embedded in the same
sentence, and, also, seldom used words and weird euphemism or paraphrases
can make the sentence hard to follow and to read it with only one
time, and making everything even harder to understand all this without
any punctuation marks makes the instructor cry and finally after
trying to correct the language, you will get boomerang, and everyone's
time has just been wasted.

Please use proofreaders before sending even your unfinished version to
the instructor and/or supervisor. You will get better comments when
they do not need first proofread your text. Moreover, they can
consentrate to the content better if the language and spelling
mistakes are not distracting the reading. Several editors have their
own proofreading tools, e.g. ispell in emacs. You can also use
Microsoft Word to proofread your thesis: it can correct also some
grammatical errors and not just misspelled words.

Note also that if you have a section or a subsection, you have to have
at least two of them, or otherwise the section or subsection title is
unnecessary. Same with the paragraphs an: you should not have sections
with only one paragraphe, and single sentence paragraph. Furthermore,
always write some text after the title before the next level title.

\section{Finding and referring to sources}

Never ever copy anything into your theses from somebody else's text
(nor your own previously published text). Never. Not even for starting
point to be rewritten later. The risk is that you forgot the copied
text to your thesis and end up to be accused of plagiarism. Plagiarism
is a serious crime in studies and science and can ruin your career
even its beginning. To repeate: never cut and paste text into your
thesis!

\subsection{Finding sources}

All work is based on someone else's work. You should find the relevant
sources of your field and choose the best of them. Also, you should
refer to the original source where a fact has been mentioned first
time. Remember source evaluation (criticism) with all sources you
find.

Good starting points for finding references in computer science are: 
\begin{itemize}
% You can use this command to set the items in the list closer to each other
% (ITEM SEParation, the vertical space between the list items) 
\setlength{\itemsep}{0pt}
\item Nelli Portal (Aalto Library): \url{http://www.nelliportaali.fi}
\item ACM Digital library: \url{http://portal.acm.org/}
\item IEEExplore: \url{http://ieeexplore.ieee.org}
\item ScienceDirect: \url{http://www.sciencedirect.com/}
\item \ldots although Google Scholar (\url{http://scholar.google.com/}) will
find links to most of the articles from the abovementioned sources, if you
search from within the university network
\end{itemize}

Some of the publishers do not offer all the text of the articles
freely, but the library has agreed on the rights to use the whole
text. Thus, you should sometimes use computers in the domain of the
university in order to get the full text. Sometimes the Nelli Portal
can also help getting the whole article instead of just the abstract.
The library has also brief instrucions how to find
information~\cite{howfindinfo}.

Instead of normal Google, use Google Scholar
(\url{http://scholar.google.fi/}). It finds academic publications whereas
normal Google find too much commercial advertisements or otherwise
biased information. Wikipedia articles should be referred to in the master
thesis only very, very seldomly. You can use Wikipedia for understanding
some basics and finding more sources, but often you cannot be sure if
the article is correct and unbiased.

One important part of the sources that you have found is the reference
list. This way you can find the original sources that all the other
research of the field refer. Often you can also find more information
with the name of the researchers that are often referred in the
articles.

\subsection{Referring to sources}

The main point in referring to sources is to separate your own
thinking and text from that of others. Facts of the research area can
be given without reference, but otherwise you should refer to
sources. This means two things: marking the source in the text where
it has been used, and listing the sources usually in the end of the
thesis in a way that help the reader to find the original source.

There are several bibliography styles, meaning how to form the
bibliography in the end of the thesis. Aalto's library has good
instructions for many styles~\cite{bibinstructions}. You should ask
from your supervisor or instructors which style you should use. This
thesis template uses the number style that is often used in software
engineering. The other style also used in the CS field,
e.g. usability, is the Harvard style where instead of numbers, the
reference is marked into the text with author's name and publishing
year. Other areas use also many other styles for making the lists and
marking the references.

In addition to the list in the end of the thesis, you have to mark the
source in the text where the source is used. There are three places
for the reference: in a sentence before the period, in the end of a
sentence after the period, or in the end of a paragraph. All of them
have different meaning. The main point is that first you paraphrase
the source using your own words and then mark the source. Next, we
give short examples that are marked with \emph{emphasised text}.

\emph{Haapasalo~\cite{HaapasaloThesis} researched database algorithms
  that allows use of previous versions of the content stored in the
  database.} This kind of marking means that this paragraph (or until
the next reference is given) is based on the source mentioned in the
beginning.  Giving the source you should use only the family name of
the first author of the article, and not give any hints about what is
the type of the article that is referred.

\emph{B+-trees offers one way to index data that is stored in to a
  database. Multiversion B+-trees (MVBT) offer also a way to restore
  the data from previous versions of the database. Concurrent MVBT
  allows many simultaneous updates to the database that is was not
  possible with MVBT.~\cite{HaapasaloThesis}} When the marking is
after the period, the reference is retrospective: all the paragraph
(or after previous reference marking) is based on the source given in
its end. If the content is very broad, you can start with saying
\emph{According to Haapasalo}, then continue referring the source with
several separate sentences, and in the end put the marking of your
source \emph{ that shows that CMVBT are the
  best. ~\cite{HaapasaloThesis}}. 

If your paragraph has several sources, the above mentioned styles are
not proper. The reader of your thesis cannot know which of your
sources give which of the statements. In this case, it is better to
use more finegraded refering where the reference markings that are
embedded in the sentences. For example, \emph{the multiversion B+-tree
  (MVBT) index of Becker et al.~\cite{becker:1996:mvbt} allows database
  users to query old versions of the database, but the index is not
  transactional.
  It's successor, the transactional MBVT (TMVBT), allows a single transaction
  running in its own thread or process to update the database concurrently
  with other transactions that only read the
  database~\cite{haapasalo:2009:tmvbt}. 
  Further development, titled the concurrent MBVT (CMVBT),
  allows several transactions to perform updates to the database at the same
  time~\cite{HaapasaloThesis}}. 
  Here, the references are marked before
  the period in the sentences where they are used.

Finally, direct quotes are allowed. However, often you should avoid
them since they do not usually fit in to your text very well. Using
direct quotes has two tricks: quotation marks and the source.  \emph{
  ``Even though deletions in a multiversion index must not physically
  delete the history of the data items, queries and range scans can
  become more efficient, if the leaf pages of the index structure are
  merged to retain optimality.''~\cite{HaapasaloThesis}} Quotes are
hard to make neatly since you should use only as much as needed
without changing the text. Moreover, you often do not really
understand what the author has mentioned with his wordings if you
cannot write the same with your own words. Remember also that never
cut and paste anything without marking the quotation marks right away,
and in general, never cut and paste anything at all!

Sometimes getting the original source can be almost impossible. In an
extremely desperate situation, you can refer with structure \emph{mr
  X~[\ldots] according to ms Y~[\ldots] defined that}, if you find a
source that refers to the original source. Note also that the
reference marking is never used as sentence element (example of how 
\textbf{not} to do it: \emph{\cite{HaapasaloThesis} describes
an optimal algorithm for indexing multiversiond databases.}).

